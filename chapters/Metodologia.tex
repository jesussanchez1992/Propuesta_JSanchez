
\newpage
\chapter{Metodolog\'ia}

El telescopio CDK 17 (corrected Dall-kirkham) es un telescopio de tubo ópticoo de diseño abierto, hecho en fibra de carbono con un paso de 43 kg, cuenta con 2 lentes de 90 mm (aplanadores de campo) y 2 espejos. 
Un espejo primario elipsoidal con una apertura de 432 mm y una relación focal de f/2,6 y un espejo secundario con una apertura de 159 mm con forma esférica.\\
Posee un enfocador Hedrik 3.5 y tres ventiladores en la parte inferior y 4 en las laterales del tubo óptico, todo esto acoplado a una montura ecuatorial Paramount ME II.\\
La montura Paramout ME II tiene un eje de contrapeso DEC 47 CM  de largo y 48 mm de diametro, 2 contrapesos de 14 Kg, un rodamiento de declinacion de 48 puntos de contacto, con una capacidad de carga de 140 Kg.


Para cumplir con los objetivos planteados se llevara la siguiente metodología.

\begin{itemize}

\item[1] Montaje del espectrógrafo con sus respectivo modulo de calibración y acople al telescopio.

\item[2] hacer el calculo de forma experimental de la rendija del espejo que permite el paso de luz al espectrógrafo.

\item[3] Realizar la captura de imágenes limpias de diferentes lamparas de emisión de laboratorio con el fin de garantizar que los elementos dispersores del espectrógrafo estén alineados con la cámara y se puedan reproducir espectros conocidos.

\item[3] Se calibrara la Montura robotizada PARAMOUNT  usando el software T-Point para garantizar un correcto apunte del telescopio al objeto de interés.



 
\end{itemize}

%%%%%%%%%%%%%%%%%%%%%%%%%%%%%%%%%%%%%%%%%%%%%%%%%%%%%%%%%%%%%%%%%%%%%%%%%%%%%%%%%%%%%%%%%%%%%%


%%%%%%%%%%%%%%%%%%%%%%%%%%%%%%%%%%%%%%%%%%%%%%%%%%%%%%%%%%%%%%%%%%%%%%%%%%%%%%%%%%%%%%%%%%%%%%
\section{Cronograma de Actividades}	
%%%%%%%%%%%%%%%%%%%%%%%%%%%%%%%%%%%%%%%%%%%%%%%%%%%%%%%%%%%%%%%%%%%%%%%%%%%%%%%%%%%%%%%%%%%%%%


\begin{center}
{\small
\begin{tabular}{|c|c|c|c|c|c|c|c|}
\hline

\textbf{Mes}/\textbf{Actividad}&\textbf{Act 1.1}&\textbf{Act 1.2}
&\textbf{Act 1.3}&\textbf{Act 2}&\textbf{Act 3}&\textbf{Act 4}&\textbf{Act 5}\\

\hline

Enero&$\bigotimes$&&&&&&\\

\hline

Febrero&$\bigotimes$&$\bigotimes$&&&&&\\

\hline

Marzo&&$\bigotimes$&$\bigotimes$&&&&\\

\hline

Abril&&&$\bigotimes$&&&&\\

\hline

Mayo&&&$\bigotimes$&$\bigotimes$&&&\\

\hline

Junio&&&$\bigotimes$&$\bigotimes$&&&\\

\hline
Julio&&&$\bigotimes$&$\bigotimes$&$\bigotimes$&&$\bigotimes$\\

\hline

Agosto&&&&&$\bigotimes$&$\bigotimes$&\\

\hline 

Septiembre&&&&&&&$\bigotimes$\\

\hline
Octubre&&&&&&&$\bigotimes$\\
\hline

\end{tabular}
}
\end{center}