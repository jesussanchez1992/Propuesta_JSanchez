\addcontentsline{toc}{chapter}{INTRODUCCIÓN}
\newpage
\chapter*{INTRODUCCIÓN}
\label{sec:intro}

Para el año 1912, el físico Victor Hess, mediante experimentos con globos aerostáticos y electrómetros, pudo evidenciar, que la ionización atmosférica aumenta proporcionalmente con la altitud. Había encontrado que el fondo de radiación presente, tenía un aporte significativo del espacio exterior, descubriendo así las astropartículas. Desde entonces, se han desarrollado decenas de experimentos alrededor del mundo, tanto en Tierra, como en el espacio, que intentan comprender a profundidad su origen, además de obtener información de fenómenos astrofísicos como explosiones de supernovas, kilonovas, entre otros. Adicionalmente, también se han desarrollado diversas aplicaciones en Tierra, que aprovechan la existencia de esta radiación por ejemplo, el detector Mu-Ray \cite{Mu-Ray}, el proyecto MuTe \cite{MuTe}, entre otros.\\

En esencia, el flujo de radiación medido en la superficie de la Tierra, es consecuencia principalmente, de partículas provenientes del espacio que llegan a la atmósfera e interactúan con ella. Este fenómeno está determinado por fenómenos físicos de dispersión, decaimientos y absorción. Además, para energías por debajo de $10^{15}$ eV, está modulado por el viento solar. La influencia de la actividad solar en el campo geomagnético y su relación con el funcionamiento de muchos dispositivos tecnológicos de la vida diaria, hacen que el transporte de partículas a través de la heliósfera sea un tema de gran interés en la física espacial. \\

Por tal razón, el \textit{Latin American Giant Observatory} (LAGO), ha desarrollado el programa de Clima Espacial, que busca entender la influencia del viento solar en el flujo de astropartículas, a partir de detectores de superficie. Dichos detectores pueden registrar partículas primarias de baja energía de forma indirecta, permitiendo obtener información de la actividad solar, complementando las observaciones que se realizan desde el espacio.\\

Con base en lo anterior, una de las tareas fundamentales  que se propone el programa de clima espacial de LAGO es estimar de forma precisa el flujo de partículas que llegan a nivel del suelo. Para tal fin, se ha desarrollado una secuencia de simulaciones, que tienen en cuenta el transporte de primarios en la heliosfera, el flujo de primarios a través de la magnetosfera, la llegada de primarios a la atmósfera, la formación de lluvias de secundarios o cascadas aéreas extensas (EAS), y la llegada de estos a los detectores de la colaboración.\\
%

En la actualidad, para el estudio de las EAS y su propagación a través de la atmósfera, LAGO hace uso de perfiles atmosféricos predefinidos en sofware CORSIKA. Para el caso de Bucaramanga, usa el perfil subtropical que corresponde a la región del globo en la que se ubica la ciudad. Sin embargo, se debe realizar un estudio más detallado que permita determinar la pertinencia de estos perfiles y la sensibilidad que tienen a variaciones en el flujo estimado.\\
%

Por lo anterior, el presente trabajo se concentra en el estudio del efecto que tiene la atmósfera en el flujo. Para esto, se ha desarrollado una metodología que permite la creación y uso de perfiles atmosféricos mensuales, para cualquier ubicación geográfica, dentro del código CORSIKA. En el capítulo 1, se realiza una breve revisión de los conceptos relacionados con las astropartículas y las cascadas aéreas extensas. Además, se describen algunos detalles relevantes de la estructura del código CORSIKA, tanto en los procesos de interacción como en las características de la atmósfera, y su modelado.\\

A continuación, en el capítulo 2, se presenta la forma como se estima el flujo de fondo de secundarios, y se introduce la metodología para crear perfiles atmosféricos mensuales, usando el Sistema Global de Asimilación de Datos (GDAS). En éste capítulo, se muestran los resultados de comparar el perfil atmosférico que actualmente se usa para la ciudad de Bucaramanga, contra los perfiles atmosféricos que fueron construidos para este trabajo.\\
%, evidenciando una diferencia significativa en los valores de densidad.\\

Finalmente, en el capítulo 3, se realiza un estudio detallado, de los efectos sobre el flujo de partículas, de las variaciones de densidad estimadas que fueron observadas mes a mes, y cómo estos efectos se relacionan con los cambios de temperatura, a lo largo del año. Además, se muestran las diferencias observadas en las estimaciones del flujo entre el perfil atmosférico subtropical, y los nuevos perfiles atmosféricos que fueron construidos. Esperándo con esto poder confirmar la importancia del uso de perfiles atmosféricos en cada sitio de la colaboración y completar la cadena de simulaciones de la colaboración LAGO.\\
%Estos resultados permitieron confirmar la importancia del uso de perfiles atmosféricos en cada sitio de la colaboración, y la metodología que se muestra en este trabajo, permite completar la cadena de simulaciones de la colaboración LAGO.\\
