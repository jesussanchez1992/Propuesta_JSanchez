%\addcontentsline{toc}{chapter}{Abstract}
\newpage
\chapter*{ABSCTRACT}
\label{sec:abst}

\textbf{TITLE:} CARACTERIZATION OF ATMOSPHERIC PROFILES FOR THE LAGO SIMULATION CHAIN.\\

\textbf{AUTHOR:} Jennifer Grisales Casadiegos.\\

\textbf{KEYWORDS: } Astroparticle, particle flux, atmosphere, simulation, extensive air shower.\\

One of the main objectives of the Space Weather program of the Latin American Giant Observatory (LAGO), is to study the influence of the solar activity on the secondary particle flux  variations, produced during the interaction between astroparticles with atmosphere. To this end, a chain of simulations is carried out, which estimates in detail, the Primary's development, from its entry into the Earth's atmosphere, to the Water Cherenkov Detector response. This work, complete the simulation chain, focusing their interest in to study the atmospheric effect on the secondary particle flux. To do this it developed a metodology that allows the creation and use of monthly atmospheric profiles, for any localization, in the EAS simulations within the CORSIKA code. Futhermore, the relevance to using the new monthly profiles it was checked, becouse they are abble to reproduce in the secondary particle flux, the effect of the temperature changes along the year. This allows refine the estimates made.
%en el proceso de formación de cascadas aéreas extensas (EAS), más específicamente,