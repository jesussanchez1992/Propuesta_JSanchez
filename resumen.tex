%\addcontentsline{toc}{chapter}{RESUMEN}
\newpage
\chapter*{RESUMEN}
\label{sec:resum}

\textbf{TÍTULO:} CARACTERIZACIÓN DE PERFILES ATMOSFÉRICOS PARA LA CADENA DE SIMULACIÓN DE LA COLABORACIÓN LAGO.\\

\textbf{AUTORA:} Jennifer Grisales Casadiegos.\\

\textbf{PALABRAS CLAVES: } Astropartículas, flujo de secundarios, atmósfera, simulaciones, cascadas aéreas extensas.\\

Uno de los objetivos del programa de clima espacial del \textit{Latin American Giant Observatory} (LAGO), es estudiar la influencia de la actividad solar en las variaciones del flujo de partículas secundarias, producidas durante la interacción de las astropartículas con la atmósfera. Con este fin, se realiza una cadena de simulaciones, que estima de forma detallada, el desarrollo del primario desde su ingreso a la atmósfera terrestre, hasta la respuesta en los detectores Cherenkov de agua. El presente trabajo, completa esta cadena de simulaciones, concentrándose en el estudio del efecto que tiene la atmósfera en el flujo de fondo de secundarios. Para ello, se desarrolló una metodología que permite la creación y uso de perfiles atmosféricos mensuales, para cualquier ubicación geográfica, en las simulaciones de EAS dentro del código CORSIKA. Además se demostró la pertinencia de reemplazar los modelos atmosféricos predeterminados por nuevos perfiles, basados Sistema Global de Asimilación de datos (GDAS), comprobando, que los nuevos modelos atmosféricos mensuales, son capaces de reproducir, en el flujo de secundarios, el efecto de los cambios de temperatura a lo largo del año, y permitiendo refinar las estimaciones realizadas.
%en el proceso de formación de cascadas aéreas extensas (EAS), más específicamente,